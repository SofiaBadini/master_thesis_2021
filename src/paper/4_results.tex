% !TEX root = main.tex


\section{Results} \label{results}

During the estimation process the econometrician chooses a method to fit the candidate model to the observed data, constructs the associated criterion function, and chooses the optimization algorithm to recover the vector of model parameters estimates. 

If a structural model's parameters are identified, it should be possible to consistently estimate them. However, the criterion function may be non-smooth, have multiple local minima, be discontinuous or non-differentiable. The (multidimensional) optimization problem may be further complicated by the presence of constraints, which are common in structural economic models: As an example, in the model presented here, the probabilities to experience a certain restriction regime need to sum up to 1, and the variance-covariance matrix of the payoff shocks needs to be positive definite and symmetric. 

All these features complicate both solving the optimization problem and making inferences about the estimated parameters. 

Moreover, carrying out the model estimation on a single dataset may not be particularly informative in this setting, as sampling variation would be neglected. On the other hand, estimation on many samples is time-consuming and computationally expensive, especially for a relatively complex model with a long planning horizon like the one presented here.

For these reasons, the following section focuses on investigating the performance of the chosen criterion function to find out whether the true model may be consistently estimated on the datasets previously described, while stopping short of the real estimation process. 

\subsection{Practical identification}

In practice, whether the time preference parameter(s) can be identified from the data depends on the curvature of the criterion function. Consider the problem of estimating a model with exponential discounting on observational data: If the criterion is computed as a function of $\delta$ only, while all the other parameters are set a their respective estimates, we need the criterion to exhibit a unique minimum (or maximum) between $0$ and $1$ to recover $\delta$.

Here, I mainly focus on the practical identification of the time preference parameters, the true values being $\delta = 0.95$ and $\beta = 0.8$, and set all the other parameters at their true, rather than estimated, values. This overcomes the hurdles of the real estimation process, simplifies the exposition and aids the visualization of the results. However, it represents a "best case scenario" where practical identification is not complicated by inaccurate estimates of the other model parameters. 

I use the Method of Simulated Moment (MSM), a simulation-based estimation method introduced by \textcite{McFadden1989}. The MSM estimator is the parameter vector that minimizes the weighted distance between the empirical set of moments, computed from the observed dataset, and the set of simulated moments generated by the candidate structural model. Formally, the MSM criterion function is:
\begin{equation}
\Delta(\theta) = [f - \tilde{f}(\theta)]'W^{-1}[f - \tilde{f}(\theta)],
\end{equation}
where $f$ represents the vector of empirical moments, $\tilde{f}(\theta)$ is the vector of simulated moments, and $W$ is a positive definite weighting matrix.

In this setting, the candidate structural model with exponential (hyperbolic) discounting $\theta$ coincides with the true data-generating process of the "observed" dataset with exponential (hyperbolic) discounting described in Section~\ref{simulated-data}. However, the empirical moments won't perfectly coincide with the simulated moments because of random variation in the payoff shocks, which induces variation in the agents' choice probabilities. This implies that the MSM criterion function is not necessarily minimized at the true parameter vector for a given finite sample.

The moments used for estimation are per-period choice probabilities, conditional on restriction regime, and the per-period wage profile (mean and standard deviation) of those who work, again conditional on restriction regime. The weighting matrix is a diagonal matrix where the weights are the inverse variances of the observed sample moments, computed via a bootstrapping procedure. 

\subsubsection{Model with exponential discounting.} Figure~\ref{fig:univariate-exp-main} represents the criterion function, evaluated at different values of the discount factor $\delta$, holding all the other parameters at their true value. The global evaluation exploits the theoretical bounds of the $\delta$ parameter. The same exercise is repeated for other selected parameters in Figure~\ref{fig:univariate-exp-others} and for some elements of the variance-covariance matrix of the shocks in Figure~\ref{fig:univariate-exp-shocks}. 

The criterion function displays a global minimum at $\delta = 0.949$ and has a minimum close to the true value for the parameters that enter the wage function for occupation A (return to education and return to experience in occupation A), while is relatively flatter for those parameters entering the wage of occupation B and for most of the elements and for most elements of the variance-covariance matrix of the shocks. In the latter cases, a closer look reveals that the criterion function is particularly non-smooth and displays multiple local minima, which may require fixing or imposing tight bounds on such parameters during the estimation process.

\begin{figure} 
\centering
\captionsetup{justification=centering}
\caption{Univariate Distribution of Discount Factor,\newline Model with Exponential Discounting}
\label{fig:univariate-exp-main}
\begin{subfigure}{0.5\textwidth}
\caption{Global Evaluation}
\includegraphics[width=\linewidth]{../../out/figures/univariate_distributions/time_preferences_exp_global.png}
\end{subfigure}%
\begin{subfigure}{0.5\textwidth}
\caption{Local Evaluation}
\includegraphics[width=\linewidth]{../../out/figures/univariate_distributions/time_preferences_exp_local.png}
\end{subfigure}

\bigskip
\footnotesize
\raggedright
\textit{\textbf{Notes}: Evaluations of Method of Simulated Moment's criterion function for model with exponential discounting based on \textcite{KeaneWolpin1994}. The criterion function is evaluated for different values of the discount factor, while all the other parameters are fixed at their true values.}
\end{figure}

\begin{figure}
\centering
\captionsetup{justification=centering}
\caption{Univariate Distribution of Wage Parameters,\newline Model with Exponential Discounting}
\label{fig:univariate-exp-others}
\begin{subfigure}{\textwidth}
\caption{Occupation A}
\includegraphics[width=\linewidth]{../../out/figures/univariate_distributions/occ_a_exp.png}
\end{subfigure}
\begin{subfigure}{\textwidth}
\caption{Occupation B}
\includegraphics[width=\linewidth]{../../out/figures/univariate_distributions/occ_b_exp.png}
\end{subfigure}

\bigskip
\footnotesize
\raggedright
\textit{\textbf{Notes}: Evaluations of Method of Simulated Moment's criterion function for model with exponential discounting based on \textcite{KeaneWolpin1994}. All parameters besides the x-axis parameter are fixed at their true values.}
\end{figure}

\begin{figure} 
\centering
\captionsetup{justification=centering}
\caption{Univariate Distribution of Variance-Covariance Matrix of Shocks,\newline Model with Hyperbolic Discounting}
\label{fig:univariate-exp-shocks}
\includegraphics[width=\linewidth]{../../out/figures/univariate_distributions/shocks_sdcorr_exp.png}

\bigskip
\footnotesize
\raggedright
\textit{\textbf{Notes}: Evaluations of Method of Simulated Moment's criterion function for model with hyperbolic discounting based on \textcite{KeaneWolpin1994}. All parameters besides the x-axis parameter are fixed at their true values.}
\end{figure}

\subsubsection{Model with hyperbolic discounting.} Figure~\ref{fig:univariate-hyp-main} represents the criterion function evaluated at different values of the time preference parameters $\beta$ and $\delta$, with all the other parameters are fixed at their true value. The same exercise is repeated for other selected parameters and elements of the variance-covariance matrix of the shocks (Figure~\ref{fig:univariate-hyp-others} and Figure~\ref{fig:univariate-hyp-shocks} in the Appendix). The results are similar to those discussed above.

\begin{figure}[!t]
\centering
\captionsetup{justification=centering}
\caption{Univariate Distribution of Time Preference Parameters,\newline Model with Hyperbolic Discounting}
\label{fig:univariate-hyp-main}
\begin{subfigure}{\textwidth}
\caption{Global Evaluation}
\includegraphics[width=\linewidth]{../../out/figures/univariate_distributions/time_preferences_hyp_global.png}
\end{subfigure}
\begin{subfigure}{\textwidth} 
\caption{Local Evaluation}
\includegraphics[width=\linewidth]{../../out/figures/univariate_distributions/time_preferences_hyp_local.png}
\end{subfigure}

\bigskip
\footnotesize
\raggedright
\textit{\textbf{Notes}: Evaluations of Method of Simulated Moment's criterion function for model with hyperbolic discounting based on \textcite{KeaneWolpin1994}. The criterion function is evaluated for different values of the discount factor (figures on the left) 
and of the present bias (figures on the right) respectively, while all the other parameters are fixed at their true values.}
\end{figure}

Figure~\ref{fig:heatmap} plots the criterion function for many combinations of $\beta$ and $\delta$, making the optimization problem two-dimensional. The darker the area, the lower the value of the criterion function associated with the corresponding combination of $\beta$ and $\delta$. All the other parameters are again held at their true values.

\begin{figure}[!t]
\centering
\captionsetup{justification=centering}
\caption{Heatmap Criterion for Time Preference Parameters,\newline Model with Hyperbolic Discounting}
\label{fig:heatmap}
\begin{center}
\includegraphics[width=\linewidth]{../../out/figures/heatmap/heatmap.png}
\end{center}
\footnotesize
\raggedright
\textit{\textbf{Notes}: Evaluations of Method of Simulated Moment's criterion function for model with hyperbolic discounting based on \textcite{KeaneWolpin1994}, for different value of discount factor and present bias.}
\end{figure}

The global minimum is at $\delta = 0.948$ and $\beta = 0.83$ and there is no basin around the minimum. The criterion function takes similar values for combinations of the time preference parameters where $\beta$ is underestimated with respect to the true value and $\delta$ is overestimated, and vice versa. The range of values of $\delta$ associated with a relatively low value of the criterion function is tighter than the range of values of $\beta$. 

This pattern is similar to the "banana-shaped through" found in \textcite{Abbring2018} and \textcite{LaibsonRepetto2007}, which consider models that are different from the one presented here. In particular, \textcite{Abbring2018} use a simple model of binary choice based on \textcite{Rust1987} to assess whether their exclusion restrictions are sufficient to achieve identification in a simulated dataset. The estimation routine is based on a minimum-distance estimator. The data are observed only for three periods, which is the theoretical minimum number of periods required for identification, while the number of simulated agents is large (1 million). 

\textcite{LaibsonRepetto2007} estimate a complex model of labor supply using data on retirement, credit card borrowing, wealth accumulation and consumption-income co-movement via the Method of Simulated Moments. 
They note that the criterion function is strictly convex in the region they consider ($0.2 < \beta < 1$ and $0.93 < \delta < 0.99$) and rises whenever both the two parameters rise or fall, a description that fits Figure~\ref{fig:heatmap}.

When $\beta$ is fixed at 1, the criterion function attains the lowest value for $\delta = 0.938$, which is consistent with the intuition that a misspecified exponential model fitted on hyperbolic dataset needs a lower discount factor to rationalize the data. 

Figure~\ref{fig:heatmap} points at the practical difficulty of disentangling $\beta$ and $\delta$ in estimation which is mentioned elsewhere in the literature. A relevant question is then whether different combinations of the two parameters produce similar life-cycle patterns and similar counterfactuals.

The following sections focus on the specification with hyperbolic discounting, first comparing the life-cycle pattern observed for unrestricted agents to that predicted by various models where the time preferences are misspecified, and then exploring the sensitivity of counterfactual predictions to model misspecification. 

\subsection{Predicted life-cycle pattern}

\begin{figure}[!t]
\centering
\captionsetup{justification=centering}
\caption{Life-Cycle Pattern, True Model vs. Misspecified Models}
\label{fig:model-fit-1}
\includegraphics[width=\linewidth]{../../out/figures/counterfactual/fit_1.png}

\bigskip
\footnotesize
\raggedright
\textit{\textbf{Notes}: Average share of individuals in each occupation in each period, as predicted by different models. The average is computed over 100 datasets, each with 10,000 agents observed for 40 periods.}
\end{figure}

\begin{figure}[!t]
\ContinuedFloat
\centering
\captionsetup{justification=centering}
\caption{Life-Cycle Pattern, True Model vs. Misspecified Models (continued)}
\label{fig:model-fit-2}
\includegraphics[width=\linewidth]{../../out/figures/counterfactual/fit_2.png}

\bigskip
\footnotesize
\raggedright
\textit{\textbf{Notes}: Average share of individuals in each occupation in each period, as predicted by different models. The average is computed over 100 datasets, each with 10,000 agents observed for 40 periods.}
\end{figure} 

Figure~\ref{fig:model-fit-1} compares the average choice probabilities pattern predicted by the true hyperbolic model and by the misspecified exponential model respectively, in absence of choice restrictions. The dotted line shows the choice probabilities for the unrestricted agents in the "observed" data.

The average pattern predicted by the true hyperbolic model is shown in Figure~\ref{fig:model-fit-1} . The share of agents in education drops quickly in the first few periods (0.668 in $t=0$, 0.431 in $t=1$, 0.251 in $t=2$, 0.156 in $t=2$). No more than 4\% and 1\% of the agents in each period chose education after $t=10$ and $t=20$ respectively.
The share of individuals in occupation B grows slowly and constantly during the life-cycle, while the share of individuals in occupation A grows quickly in the first 5 observed periods (from 20\% to nearly 64\%), peaks between the 10th and the 15th period and then declines steadily. 
The share of agents who choose home increases during the first period (from 10\% to nearly 17\%), declines steadily until the 25th period, and increases again at the end of the life-cycle in what \textcite{KeaneWolpin1994} characterize as "voluntary retirement". 

The misspecified model with exponential discounting compensates the absence of present bias with a lower discount factor, which is set at $0.938$, the value of $\delta$ that minimized the criterion function when $\beta$ is set to 1. All the other parameters are set at the true model values. With respect to the true model, the exponential one predicts a lower share of agents in education in the first 10 periods, a higher share of agents choosing occupation A in the first half of the life cycle, and a substantially similar pattern in occupation B. The share of agents choosing home, which is high and growing in the first 5 periods (from 19\% to 24\%), remains higher than the hyperbolic counterpart until the 30th period. 

Figure~\ref{fig:model-fit-2} compares the average choice probabilities predicted by the true model with those predicted by models where both $\beta$ and $\delta$ are misspecified. I choose a few combinations of present bias and discount factor values in the dark area of Figure~\ref{fig:heatmap}, which have criterion values close to that achieved by the true combination of parameters. 
 
The average predictions for occupation A and occupation B are virtually identical. The model where $\beta$ is closer to unity than in the true value ($0.86$ vs. $0.8$) while $\delta$ is lower ($0.946$ vs. $0.95$) predicts an higher share of agents at home when the home choice is peaking, around the 5th observed period, and a lower share of agents in education in the same interval. Vice versa, when $\beta$ is set at $0.78$ and $\delta$, to compensate, is set at $0.952$, the average share of agents at home is lower, while the share of agents in education is higher. This seems to suggest that the discount factor influences the trade-off between home and education at the beginning of the life cycle more than the present bias, which is not surprising given that the planning horizon is long and the costs of education are very low compared to the long-term benefits.


\subsection{Counterfactual predictions}

This section studies how the introduction of a 2000 USD tuition subsidy affects final and per-period experience in education. Figure~\ref{fig:ridgeplot} shows the distribution of the true subsidy effect on years of education in each period, for an example dataset. Periods are indicated on the y-axis, while the x-axis indicates the increase in years of education with respect to a counterfactual dataset where the subsidy is not implemented. All agents start with $10$ periods of education in period $0$, which is omitted. 

The figure shows that the distribution of the true subsidy effect is decidedly skewed to the right, in each period. The subsidy is mostly effective in the first 10 periods of the life cycle, which is graphically represented by the per-period distributions shifting to the right of the plot. In the last 20 periods the distributions are stacked vertically, which means that agents have substantially stopped to accumulate experience in education. 

\begin{figure}[!t]
\centering
\captionsetup{justification=centering}
\caption{Predicted Effect of Tuition Subsidy on Experience in Education}
\label{fig:ridgeplot}
\includegraphics[width=\linewidth]{../../out/figures/counterfactual/ridgeplot.png}

\bigskip
\footnotesize
\raggedright
\textit{\textbf{Notes}: Effect of 2000 USD tuition subsidy on experience in education, in each period, as predicted by the true model ($10,000$ agents, $40$ observed periods).}
\end{figure}

The distribution in period 39 (last row of Figure~\ref{fig:ridgeplot} can be collapsed to single numbers if the quantity of interest is simply the average change in final years of education. Table~\ref{tab:counterfactuals} compares a few summary statistics derived from the distribution of the tuition subsidy effect on final years of education, as predicted by the true model and the misspecified models. The distribution is generated by 100 simulated datasets with sampling variation, and the standard deviation should be interpreted as reflecting the simulation error. 

The true model with hyperbolic discounting predicts that agents spend on average about $1.07$ additional years in education. The median is lower ($0.99$) because of the distribution's right tail, as seen in Figure~\ref{fig:ridgeplot}. The model with exponential discounting predicts an increase of $0.78$ years, with a tighter standard deviation. The average predicted effect is lower than the average true model's effect whenever the discount factor is lower than $0.95$, despite the present bias being higher than $0.8$, and vice versa when the discount factor is higher than $0.95$. Again, the discount factor, rather than the present bias, appears to be the main driver of educational decisions.

\begin{table}[!t]
\centering
\caption{Effect of tuition subsidy on average years of education} \label{tab:counterfactuals}
\begin{adjustbox}{max width=\textwidth}
\input{../../out/tables/counterfactuals}
\end{adjustbox}
\end{table}

Figure~\ref{fig:counterfactuals} shows the average effect of the tuition subsidy on experience in education, in each period, again computed on 100 datasets with sampling variation. It confirms that, in all models, the tuition subsidy is associated with the largest increase in years of education at the beginning of the life cycle, and that the lower the discount factor the lower the average predicted effect in each period. The shaded areas visually represent the simulation error, which is largely overlapping whenever the model is hyperbolic and smaller for the model with exponential discounting. The latter feature can be explained by the decision rule of naïve agents being dynamically inconsistent and more sensitive to random payoff shocks.

\begin{figure}[!t]
\centering
\captionsetup{justification=centering}
\caption{Counterfactual Predictions, True Model vs. Misspecified Models}
\label{fig:counterfactuals}
\begin{subfigure}{\textwidth}
\includegraphics[width=\linewidth]{../../out/figures/counterfactual/prediction_1.png}
\end{subfigure}
\begin{subfigure}{\textwidth}
\includegraphics[width=\linewidth]{../../out/figures/counterfactual/prediction_2.png}
\end{subfigure}

\bigskip
\footnotesize
\raggedright
\textit{\textbf{Notes}: Effect of tuition subsidy of 2000 USD on average years of education in each period, predicted by true model and misspecified models. The average is computed over 100 datasets, each with 10,000 agents observed for 40 periods.}
\end{figure}

\textcite{KempterTolan2018} estimate both an exponential and an hyperbolic model on the same German Socio-Economic Panel data and similarly, when studying the effect of a student grant on successful years of education, find that the behavioral adjustment of hyperbolic discounters is larger. In their educational choice model hyperbolic discounters care less about achieving successful years of education with respect to exponential discounters, and are more likely to drop out shortly before achieving a degree. Therefore, hyperbolic discounters realize higher gains from staying in education longer. 

In the model presented here, there is no distinction between successful and actual years of education. The trade-off for hyperbolic discounters is initially between home and education and, later, between education and occupation A. The simulated data show that hyperbolic discounter are more likely to choose the home alternative early in their life cycle: A tuition subsidy draws to education those at the margin. 
Unsurprisingly, agents react to the introduction of a tuition subsidy and to the introduction of a (deterministic) restriction in their choice set in the same way, that is, spending more time in education. Indeed, a choice restriction raises the option value of education, while the tuition subsidy lowers its per-period cost. 