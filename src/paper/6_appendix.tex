% !TEX root = main.tex

\begin{appendices} \label{appendix}

\setlength\tabcolsep{2pt}
\begin{longtable}{lllll}
\caption{Model with Exponential Discounting -- Parametrization} \\ 
\hline
\textbf{Parameter}					 	& \textbf{Value} &			\textbf{Comment} \\
\midrule
\textit{Time preference parameters} 	&		         &          \\
\midrule
$\delta$								&		    0.95 &			Discount factor \\
$\beta$									&              1 &			Present bias \\
\addlinespace 							
\midrule
\textit{Occupation A}                   &                &          \\
\midrule
$\alpha^A_{0}$ 							&              8 &          Log of rental price \\
$\alpha^A_{1}$ 							&           0.07 &          Return to an additional year of schooling \\
$\alpha^A_{2}$ 							&          0.055 &          Return to same sector experience \\
$\alpha^A_{3}$ 						    &              0 &          Return to same sector, quadratic experience \\
$\alpha^A_{4}$ 							&              0 &          Return to other sector experience \\
$\alpha^A_{5}$ 						    &              0 &          Return to other sector, quadratic experience \\
\addlinespace
\hline
\textit{Occupation B}                   &                &          \\
\hline
$\alpha^B_{0}$ 							&         	 7.9 &          Log of rental price \\
$\alpha^B_{1}$							&        	0.07 &          Return to an additional year of schooling \\
$\alpha^B_{2}$	 						&        	0.06 &          Return to same sector experience \\
$\alpha^B_{3}$	 						&              0 &          Return to same sector, quadratic experience \\
$\alpha^B_{4}$	 						&          0.055 &          Return to other sector experience \\
$\alpha^B_{5}$	 						&              0 &          Return to other sector, quadratic experience \\
\addlinespace
\hline
\textit{Education}						&                &          \\
\hline
$\gamma_0$								&           5000 &          Constant reward for choosing education \\
$\gamma_1$								&          -5000 &          Cost of going to college (tuition, etc.) \\
$\gamma_2$				    			&         -20000 &          Penalty for going back to school \\
\addlinespace
\hline
\textit{Home}                           &                &          \\
\hline
$\Theta$								&          21500 &          Constant reward of home alternative \\
\addlinespace
\hline
\pagebreak
\textit{Standard Deviation/Correlation Matrix}           &          & \\
\hline
$\sigma_0$ 								&              1 &  		Standard deviation for A \\
$\sigma_1$  						    &              1 &  		Standard deviation for B \\
$\sigma_2$ 								&           7000 &  		Standard deviation for E \\
$\sigma_3$ 	                            &           8500 &  		Standard deviation for H \\
$\rho_{1,0}$							&            0.5 &  		Correlation between B and A \\
$\rho_{2,0}$						    &              0 &  		Correlation between E and A \\
$\rho_{2,1}$							&              0 &  		Correlation between E and B\\
$\rho_{3,0}$							&          	    0 &  		Correlation between H and A \\
$\rho_{3,1}$ 							&              0 &  		Correlation between H and B \\
$\rho_{3,2}$							&           -0.5 &  		Correlation between H and E \\
\addlinespace
\hline
\textit{Probabilities}                  &                &          \\
\hline
$\pi_R$ 								&         	 0.5 &     		Probability of experiencing choice restriction on B \\
$\pi_{UNR}$							    &    	     0.5 & 		 	Probability of not experiencing choice restriction \\
$\pi_{E,t-1}$						    &              1 &  		Probability that the first lagged choice is education \\
$\pi_{E=10,t=0}$					    &              1 &  		Probability that the initial level of education is $10$ \\
\hline
\label{tab:parametrization}
\end{longtable}
\footnotesize
\justify
\textit{\textbf{Notes}: The parametrization of the model with hyperbolic discounting is nearly identical: $\beta$ is set to $0.8$ and $\pi_{UNR} = 0.34, \pi_R = \pi_{VR} = 0.33$, where $\pi_{VR}$ is the probability of facing choice restrictions on both A and B.}





\begin{figure} 
\centering
\captionsetup{justification=centering}
\caption{Univariate Distribution of Wage Parameters,\newline Model with Hyperbolic Discounting}
\label{fig:univariate-hyp-others}
\begin{subfigure}{\textwidth}
\caption{Occupation A}
\includegraphics[width=\linewidth]{../../out/figures/univariate_distributions/occ_a_hyp.png}
\end{subfigure}
\begin{subfigure}{\textwidth}
\caption{Occupation B}
\includegraphics[width=\linewidth]{../../out/figures/univariate_distributions/occ_b_hyp.png}
\end{subfigure}

\bigskip
\footnotesize
\raggedright
\textit{\textbf{Notes}: Evaluations of Method of Simulated Moment's criterion function for model with hyperbolic discounting based on \textcite{KeaneWolpin1994}. All parameters besides the x-axis parameter are fixed at their true values.}
\end{figure}


\begin{figure} 
\centering
\captionsetup{justification=centering}
\caption{Univariate Distribution of Variance-Covariance Matrix of Shocks,\newline Model with Hyperbolic Discounting}
\label{fig:univariate-hyp-shocks}
\includegraphics[width=\linewidth]{../../out/figures/univariate_distributions/shocks_sdcorr_exp.png}

\bigskip
\footnotesize
\raggedright
\textit{\textbf{Notes}: Evaluations of Method of Simulated Moment's criterion function for model with hyperbolic discounting based on \textcite{KeaneWolpin1994}. All parameters besides the x-axis parameter are fixed at their true values.}
\end{figure}

\end{appendices}




