% !TEX root = main.tex

\section{Conclusion} \label{conclusion}

This thesis assesses the practical identification of the time preference parameters in a discrete choice dynamic model of occupational choice, after introducing empirically-motivated exclusion restrictions that influence the size of the agents' choice set. In particular, agents may or may not face future employment restrictions that depend deterministically on their educational choices. The intuition is that comparing the behavioral response of similar agents to different (expected) futures may reveal information on their time preferences. 

In the literature, both theoretical arguments for identification and empirical identification strategies exploit, with mixed results, variables that leave the per-period utility function unaffected, while being relevant to the agents' decisions. Formal identification is elusive for the general case of a partially naive hyperbolic discounter, may not lead to consistently estimated parameters in practice, and often requires assumptions that may be too restrictive in empirical applications. On the other hand, empirical identification often requires the data to have specific features, or additional parametric assumptions about certain components of the model.

Here, practical identification is studied on two simulated datasets, generated respectively by two models of occupational choice based on \textcite{KeaneWolpin1994}, via the Method of Simulated Moments. In the first dataset, agents are exponential discounters, therefore their attitude towards the future is solely described by their discount factor, $\delta$. In the second dataset, the agents are hyperbolic discounters and are completely naive with respect to their own time preferences. Their time preferences are described by the discount factor and the present bias, $\beta$.
 
I simplify the problem by focusing only on the practical identification of the time preference parameters, while all the other parameters are set to their true values. Additional results repeat the exercise for other selected parameters, that enter either the wage functions or the variance-covariance matrix of the payoff shocks. This approach has an obvious limitation: It does not address the impact on identification of additional issues that may occur during the estimation process, such as the inaccurate estimates of the other parameter. 

When the problem is one-dimensional, that is, the criterion function is computed as a function of $\delta$ or $\beta$ only, the criterion appears to be smooth and displays a (global) minimum close to the true value of the time preference parameters for both simulated datasets, which is an encouraging, although preliminary, result. 

For the second dataset, I address the two-dimensional problem of recovering $\delta$ and $\beta$ simultaneously, which is visualized by means of a heat map of the criterion function. The combination of parameters that jointly minimize the criterion function slightly underestimates the discount factor and overestimates the present bias. Interestingly, the pattern around the minimum resembles the "banana-shaped through" that appears elsewhere in the literature: The values of the criterion function are similar for combination of parameters that underestimate the discount factor and overestimate the present bias, and vice versa. This suggests difficulties in disentangling the two parameters during estimation.

Do combinations of time preference parameters belonging to the "banana-shaped through" generate significantly different predictions? Again assuming that all other parameters are fixed at their true values, I compare an arbitrary counterfactual outcome and the life-cycle patterns predicted by the "true" model (that generates the second dataset) and a few misspecified versions of the same model. I additionally include the misspecified model with exponential discounting which comes closer to matching the data. When the present bias is fixed at $1$, that is, the corner case equivalent to exponential discounting, the discount factor that minimizes the criterion function is lower than the "true" $\delta$, which is consistent with economic intuition.

The results show that all the models with hyperbolic discounting considered predict, on average, a substantially similar life-cycle pattern. The counterfactual predictions for the effect of a tuition of subsidy on years of education are also overlapping. The model with exponential discounting predicts more agents at home and in occupation A, and fewer agents choosing education. Moreover, the behavioral adjustment of exponential discounters to the introduction of a subsidy is smaller than that of hyperbolic discounters, which is again consistent with economic intuition. Finally, the discount factor appears to be more relevant for educational choice than the present bias.

This work has some obvious limitations, which may be overcome with higher computational power: Besides bypassing the estimation procedure and related issues, practical identification is investigated n only on two simulated datasets and on a few selected parameters. Therefore, the results should be seen as partial and preliminary. 
Nevertheless, these partial results appear to be consistent with economic intuition, and similar patterns are observed elsewhere in the literature on the identification of time preferences. Further research on the sensitivity of model predictions to misspecification may be useful, as estimating the parameters separately seems to be a recurring problem in empirical research. 